%% start of file `template.tex'.
%% Copyright 2006-2010 Xavier Danaux (xdanaux@gmail.com).
%
% This work may be distributed and/or modified under the
% conditions of the LaTeX Project Public License version 1.3c,
% available at http://www.latex-project.org/lppl/.


\documentclass[14pt,a4paper]{moderncv}

% moderncv themes
\moderncvtheme[blue]{classic}                 % optional argument are 'blue' (default), 'orange', 'red', 'green', 'grey' and 'roman' (for roman fonts, instead of sans serif fonts)
%\moderncvtheme[green]{classic}                % idem
\usepackage[T2A]{fontenc}


% character encoding
\usepackage[utf8]{inputenc}                   % replace by the encoding you are using
\usepackage[russian]{babel}
\renewcommand{\rmdefault}{cmr} % Шрифт с засечками
\renewcommand{\sfdefault}{cmss} % Шрифт без засечек
\renewcommand{\ttdefault}{cmtt} % Моноширинный шрифт

% adjust the page margins
\usepackage[scale=0.8]{geometry}
%\setlength{\hintscolumnwidth}{3cm}						% if you want to change the width of the column with the dates
%\AtBeginDocument{\setlength{\maketitlenamewidth}{6cm}}  % only for the classic theme, if you want to change the width of your name placeholder (to leave more space for your address details
%\AtBeginDocument{\recomputelengths}                     % required when changes are made to page layout lengths

% personal data
\familyname{\newline Папченко}
\firstname{Владимир}


%\title{Resumé title (optional)}               % optional, remove the line if not wanted
%\address{street and number}{postcode city}    % optional, remove the line if not wanted
\mobile{89170169283 (Telegram, WhatsApp, Viber)}                    % optional, remove the line if not wanted
%\phone{phone (optional)}                      % optional, remove the line if not wanted
%\fax{fax (optional)}                          % optional, remove the line if not wanted
\email{v.e.papchenko@gmail.com}                      % optional, remove the line if not wanted
%\homepage{homepage (optional)}                % optional, remove the line if not wanted
\extrainfo{skype papchenkove} % optional, remove the line if not wanted
%\photo[64pt]{picture}                         % '64pt' is the height the picture must be resized to and 'picture' is the name of the picture file; optional, remove the line if not wanted
%\quote{Some quote (optional)}                 % optional, remove the line if not wanted

% to show numerical labels in the bibliography; only useful if you make citations in your resume
\makeatletter
\renewcommand*{\bibliographyitemlabel}{\@biblabel{\arabic{enumiv}}}
\makeatother

% bibliography with mutiple entries
%\usepackage{multibib}
%\newcites{book,misc}{{Books},{Others}}

\nopagenumbers{}                             % uncomment to suppress automatic page numbering for CVs longer than one page
%----------------------------------------------------------------------------------
%            content
%----------------------------------------------------------------------------------
\begin{document}
\maketitle

\section{Личные данные}
\cventry{ФИО}{Папченко Владимир Евгеньевич}{}{}{}{}
\cvline{Дата рождения}{14 ноября 1986 года}
\cvline{Пол}{мужской}
\cvline{Семейное положение}{холост}
\cvline{Вредные привычки}{отсутствуют}

\section{Образование}
\cventry{2009--2012}{Аспирант}{Самарский Государственный Университет Путей Сообщения, 05.13.01 Системный анализ, управление и обработка информации}{Самара}{Тема дисертации: "Параметрическая идентификация многомерных по входу нелинейных динамических систем при наличии помех наблюдений в выходном сигнале"}{Область научных интересов: системный анализ, параметрическая идентификация}
\cventry{2004--2009}{Инженер (диплом с отличием)}{Самарский Государственный Университет Путей Сообщения, Информационные системы и технологии}{Самара}{Тема диплома:}{Система контроля состояния информационных ресурсов}  % arguments 3 to 6 can be left empty
\cventry{2005--2008}{Специалист (диплом с отличием)}{Самарский Государственный Университет Путей Сообщения, Переводчик в сфере профессиональной коммуникации}{Самара}{}{Дополнительное профессиональное образование}

\section{Навыки}
\cventry{Языки программирования}{ABAP, Java, Delphi, PL/SQL, JavaScript}{}{}{}{}
\cventry{СУБД}{Oracle, MS SQL Server}{}{}{}{}
\cventry{Web}{JSP, JSTL, Spring MVC, Velocity, HTML, CSS}{}{}{}{}
\cventry{ОС}{Windows, Debian, SLES}{}{}{}{}
\cventry{Web-серверы}{Apache, Apache Tomcat}{}{}{}{}
\cventry{VCS}{Subversion, Git}{}{}{}{}
\cventry{RAD}{Eclipse, Borland Delphi, MS Visual Studio}{}{}{}{}
\cventry{Офисные продукты}{MS Word, Excel, Access, Outlook на уровне разработчика}{}{}{}{}
\cventry{Научные приложения}{MATLAB, MathCAD, LATEX}{}{}{}{}

\section{Иностранные языки}
\cvlanguage{Английский}{Intermediate}{}

\section{Опыт работы}
%\subsection{Vocational}
\cventry{апрель 2008--декабрь 2012}{Программист}{Самарский ИВЦ ОАО РЖД}{Самара}{}{%
Проекты:%
\begin{itemize}%
\item Система учета занятости учебных классов.
  \begin{itemize}%
    \item Обязанности: разработка системы управления заявками на учебные классы.
    \item Используемые средства и технологии: Java, Oracle 10g, Spring MVC, Apache Velocity, Apache Tomcat.
  \end{itemize}
\item Средства для построения отчетов из HP Open View Service Desk.
  \begin{itemize}%
    \item Обязанности:
      \begin{itemize}%
        \item Разработаны модули получения данных из HP Open View Service Desk.
        \item Разработано web-приложение для анализа ключевых показателей работы предприятия.
        \item Разработаны приложения для построения отчетов в MS Excel.
    \end{itemize}
    \item Используемые средства и технологии: Java, Oracle 10g, Spring MVC, Apache Tomcat, Apache POI.
  \end{itemize}
\item Сайт службы поддержки пользователей.
  \begin{itemize}%
    \item Обязанности: доработка, сопровождение, администрирование.
    \begin{itemize}%
        \item Разработан модуль интеграции с HP Open View Service Desk.
        \item Разработан модуль интеграции с системой заявок.
        \item Разработан модуль отображения информации о текущих простоях систем.
    \end{itemize}
    \item Используемые средства и технологии: PHP, MS SQL Server, Java, Apache Ant.
  \end{itemize}
\item Утилиты для автоматизации работы с HP Open View Service Desk.
  \begin{itemize}%
    \item Обязанности: разработка приложений для автоматизации рутинных операций с HP Open View Service Desk
      \begin{itemize}%
        \item Разработан модуль генерации плановых нарядов.
        \item Разработан модуль автоматического закрытия выполненных работ.
      \end{itemize}
    \item Используемые средства и технологии: Java, Apache Ant.
  \end{itemize}
\item АС Карта рабочих групп.
  \begin{itemize}%
    \item Обязанности: разработка и проектирование приложения для поиска рабочих групп диспетчером службы поддержки.
    \item Используемые средства и технологии: Oracle 10g, Delphi 7.
  \end{itemize}
\item[] Также в обязанности входило: сопровождение и доработка АСУ, администрирование СУБД (Oracle 9-11), веб-серверов (Apache, Apache Tomcat), ОС (W2k3, SLES 11); разработка средств автоматизации администрирования, мониторинга работоспособности систем.
  \begin{itemize}
    \item[]
  \end{itemize}
\end{itemize}}

\cventry{январь 2013--н.в.}{Ведущий инженер}{ООО Газпром информ}{Самара}{}{%
\begin{itemize}
  \item Обязанности: разработка приложений, интерфейсов, отчетов на языке ABAP для модулей SAP ERP FI, FI-AA, RE-FX, FI-SL, PS, расширение стандартной функциональности, разработка модулей интеграции со смежными системами.
  \item Технологии: ALV Grid, BAPI, BADI, Batch Input, XSLT, ABAP Objects, user-exit, Enhancements, PDF-формуляры, XI/PI.
\end{itemize}
}

\section{Увлечения}
\cvline{}{\small Интеллектуальные игры, чтение, туризм}

\section{Награды, сертификаты, портфолио, проекты}

% \cvline{hobby 2}{\small Description}
% \cvline{hobby 3}{\small Description}

\section{Extra 1}
\cvlistitem{Item 1}
\cvlistitem{Item 2}
% \cvlistitem[+]{Item 3}            % optional other symbol

% \renewcommand{\listitemsymbol}{-} % change the symbol for lists

% \section{Extra 2}
% \cvlistdoubleitem{Item 1}{Item 4}
% \cvlistdoubleitem{Item 2}{Item 5 \cite{book1}}
% \cvlistdoubleitem{Item 3}{}

% Publications from a BibTeX file without multibib\renewcommand*{\bibliographyitemlabel}{\@biblabel{\arabic{enumiv}}}% for BibTeX numerical labels
%\nocite{*}
%\bibliographystyle{plain}
%\bibliography{publications}       % 'publications' is the name of a BibTeX file

% Publications from a BibTeX file using the multibib package
%\section{Publications}
%\nocitebook{book1,book2}
%\bibliographystylebook{plain}
%\bibliographybook{publications}   % 'publications' is the name of a BibTeX file
%\nocitemisc{misc1,misc2,misc3}
%\bibliographystylemisc{plain}
%\bibliographymisc{publications}   % 'publications' is the name of a BibTeX file

\begin{thebibliography}{99}
\bibitem{book1}
Кацюба О.А., Волныкин А.Н. Идентификация многомерных по входу стационарных линейных динамических систем // Известия Самарского научного центра Российской академии наук. - 2006. – С. 1026-1033.
\bibitem{book2}
Кацюба О.А., Жданов А.И. Идентификация методом наименьших квадратов параметров уравнений авторегрессии с аддитивными ошибками измерений// Автоматика и телемеханика. - 1982. - №2. – с. 29-38 .
\bibitem{book3}
Stoica P., Soderstrom T. Bias correction in least-squares identification// Int. J. Control. - 1982. - v. 35, № 3. - p. 452.
\end{thebibliography}

\end{document}


%% end of file `template_en.tex'.
